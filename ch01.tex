\chapter{Informational Patterns}
\label{ch01}

\section{Introduction}
\label{ch01.sec.1}

The collection of patterns identified here are loosely categorized as those that serve information-sharing and manipulation purposes. The Information Resource pattern itself really needs no motivation. It has an established reference implementation called The World Wide Web\footnote{Or, these days, ``The Web''.}. The logically-named, interlinked HTML documents became \emph{The Way} of sharing information with collaborators, partners, clients and the general public. As we grew discontent with static documents, we found ways to make them dynamic and to reflect the contents of our data stores in user-friendly and navigable ways. This in term fueled a hunger for zero installation dynamic applications through the \emph{Universal Client}, a.k.a. the browser.

This migration from static documents to dynamic applications certainly involved new technologies (e.g. AJAX, JavaScript libraries, etc.), but it was enabled based on the underlying pieces of logically-named \emph{Uniform Resource Locators} (URLs), the \emph{Hypertext Transfer Protocol} (HTTP) and \emph{Hypertext Markup Language} (HTML).

These technologies combined quite seamlessly to bring us a platform usable by scientists, knowledge workers, grandparents and children. It was not an accident that this happened. Tim Berners-Lee, Roy Fielding, Dan Connolly, Dave Raggett and others from the \emph{Internet Engineering Task Force} (IETF) and \emph{World Wide Web Consortium} (W3C) toiled to create a platform unencumbered by patents but displaying the architectural properties of loose-coupling, scalability, flexibility and extensibility.

The part that still requires a surprising amount of motivation is to think about these technologies within an organization's firewalls as well. We forget that Tim's original proposal was written in the context of solving the information sharing needs of a single (albeit complex) organization, the European Organization for Nuclear Research (CERN).

In his storied proposal\footnote{Deemed ``vague but exciting'' by his supervisor}, he said:

\begin{quote}
If a CERN experiment were a static once-only development, all the information could be written in a big book.\citet{bernerslee89}
\end{quote}

Clearly this was not the case, however. In trying to do fundamental science, new participants, new strategies, new results and new experiments needed to be accepted constantly. Given that CERN brought researchers from around the world and interacted with other academic, commercial and governmental institutions, no social or organizational policies could force standardization amongst all the participants. A technological solution that embraced change was proposed. After the proposal got no traction for quite some time, Tim was eventually given permission to work on it on the side. The rest is history.

Most of the world embraced the vision whole-heartedly because it allowed them to participate regardless of the technology choices they had made. Different operating systems, platforms, etc. were melded into an interoperable platform where anyone could contribute. Broken links were tolerated and localized to specific interactions. The decentralized approach of logically-connected resources allowed demand to spike unpredictably.

All of this is to say that the approach looked at the world of information integration differently. Change was expected. Breakage was expected. Consensus was not required. Central authority could be embraced or not.

\newpage
\section{Information Resource}
\label{ch01.sec.2}

\subsection{Intent}
\label{ch01.sec.2.intent}
The Information Resource pattern is the basic building block of the Web. It provides an addressable, resolvable relatively course-grained piece of information. It should generally have a stable identifier associated with it and optionally support content negotiation, updates and deletions.

\subsection{Motivation}
\label{ch01.sec.2.motivation}


\subsection{Consequences}
\label{ch01.sec.2.consequences}

Document
Data
Service

\newpage
\section{Collection Resource}
\label{ch01.sec.3}

\subsection{Intent}
\label{ch01.sec.3.intent}
The Collection Resource is a form of the Information Resource pattern that allows arbitrary partitioning of the information space. To keep client resources from having to know too much about the server layout, these resources should provide the mechanism for pagination across the collection.

\subsection{Motivation}
\label{ch01.sec.3.motivation}

\subsection{Consequences}
\label{ch01.sec.3.consequences}

Partition
Pagination

\newpage
\section{Non-Information Resource}
\label{ch01.sec.4}

\subsection{Intent}
\label{ch01.sec.4.intent}
The Non-Information Resource is a basic building block of the Semantic Web. It provides an addressable, resolvable identifier representing a resource that is not otherwise directly accessible. It can be used for people, organizations, concepts or any term or relationship from a domain of interest.

\subsection{Motivation}
\label{ch01.sec.4.motivation}

\subsection{Consequences}
\label{ch01.sec.4.consequences}

Concept

\newpage
\section{Named Query}
\label{ch01.sec.5}

\subsection{Intent}
\label{ch01.sec.5.intent}
The Named Query pattern generalizes the idea of providing an identifier for a reusable query. The naming process might be explicit or implicit based on ad hoc, but identifiable, query structures. By giving the query identity, it becomes reusable, shareable, cacheable and perhaps the source of a new data source.

\subsection{Motivation}
\label{ch01.sec.5.motivation}

\subsection{Consequences}
\label{ch01.sec.5.consequences}

SPARQL Endpoint

