\chapter{Procedural Patterns}
\label{ch03}

\section{Introduction}
\label{ch03.sec.1}

\newpage
\section{Gateway}
\label{ch03.sec.2}

\subsection{Intent}
\label{ch03.sec.2.intent}
The Gateway pattern is notionally a combination of the Information Resource, Named Query and Tranformation patterns. The distinction is that it is generally intended for a specific client use case. Rather than a server resource being established for general use, the Gateway pattern establishes a proxy for a client interested in an orchestration, data aggregation, content extraction or other processing of one or more backend sources.

\subsection{Motivation}
\label{ch03.sec.2.motivation}

\subsection{Consequences}
\label{ch03.sec.2.consequences}

ql.io

\newpage
\section{Curated URI}
\label{ch03.sec.3}

\subsection{Intent}
\label{ch03.sec.3.intent}
The Curated URI pattern establishes a stewardship commitment to the clients of a named resource. Rather than relying on the good graces of a resource provider to keep an identifier stable over time, a new identifier is chosen to be redirected toward another endpoint. The target resource can be moved as long as the redirection is maintained without having a negative impact on clients.

\subsection{Motivation}
\label{ch03.sec.3.motivation}

\subsection{Consequences}
\label{ch03.sec.3.consequences}

PURL

\newpage
\section{Heatmap}
\label{ch03.sec.4}

\subsection{Intent}
\label{ch03.sec.4.intent}
The Heatmap pattern grows out of the dual nature of an addressable resource identifier. As a name, it affords disambiguation in a global information space at a very fine granularity. As a handle, it represents the interface to the resource itself. Metadata captured about how this resource is used, accessed, failure rates, etc. represent crucial business intelligence. This information can be used to identify business opportunities, operational planning, dynamic routing and more.

\subsection{Motivation}
\label{ch03.sec.4.motivation}

\subsection{Consequences}
\label{ch03.sec.4.consequences}

\newpage
\section{Workflow}
\label{ch03.sec.5}

\subsection{Intent}
\label{ch03.sec.5.intent}
The Workflow pattern encodes a series of steps into a resource abstraction where the client learns what options are available through the resource representation. The server is in charge of enabling and disabling state transitions based on the context of the requests, the client choices and other inputs.

\subsection{Motivation}
\label{ch03.sec.5.motivation}

\subsection{Consequences}
\label{ch03.sec.5.consequences}

