
\chapter{Jordan Canonical Form}
\index{jordan canonical form@Jordan Canonical Form}%

\section{The Diagonalizable Case}

Although, for simplicity, most of our examples
will be over the real numbers
(and indeed over the rational numbers), we will consider that
\textit{all of our vectors and matrices
are defined over the complex numbers} $\mathbb{C}$.
It is only with this assumption that the
theory of Jordan Canonical Form (JCF) works
\index{jcf@JCF}%
completely.
%See Remark~1.9 for the key reason why.\\
See Remark~\ref{rem1.4} for the key reason why.\\

\begin{definition}
\label{def1.1}
If \(v \neq 0\) is a vector such that, for some
$\lambda$,
\[
A v = \lambda v~,
\]
then $v$ is an \textit{eigenvector} of $A$ associated
\index{eigenvector}%
to the \textit{eigenvalue} $\lambda$.
\index{eigenvalue}%
\end{definition}

\begin{example}
\label{exa1.2}
Let $A$ be the matrix $A =$
Then, as you can check, if $v_1 =$ 
then $A v_1 = 3 v_1$, so $v_1$ is an eigenvector of $A$ with associated
eigenvalue $3$, and if $v_2 =$ 
then $A v_2 = -2 v_2$, so $v_2$ is
an eigenvector of $A$ with associated eigenvalue $-2$.
\end{example}


We note that the definition of an eigenvalue/eigenvector can be expressed
in an alternate form.
Here $I$ denotes the identity matrix:

For an eigenvalue $\lambda$ of $A$, we let $E_{\lambda}$ denote
the \textit{eigenspace} of $\lambda$,
\index{eigenspace}%
\[
E_{\lambda}=\{v \ |\ Av=\lambda v\}=\{v \ |\ (A-\lambda I)v=0\} =
\]

(The kernel  is also known as the nullspace

We also note that this alternate formulation helps us find eigenvalues and
eigenvectors.
For if $(A - \lambda I) v = 0$ for a nonzero vector $v$,
the matrix $A - \lambda I$ must be singular,
and hence its determinant must be
$0$.
This leads us to the following definition.

\begin{definition}
\label{def1.3}
The \textit{characteristic polynomial}
\index{characteristic polynomial}%
of a matrix $A$ is the polynomial 
\end{definition}

\begin{remark}
\label{rem1.4}
This is the customary definition of the characteristic
\index{characteristic polynomial}%
polynomial.
But note that, if $A$ is an $n$-by-$n$ matrix, then the matrix
$\lambda I - A$ is obtained from the matrix $A - \lambda I$ by multiplying
each of its $n$ rows by $-1$, and hence
In practice, it is most convenient
to work with $A -\lambda I$ in finding eigenvectors---this minimizes
arithmetic---and when we come to find
chains of generalized eigenvectors in
Section~1.2,
it is (almost) essential to use $A -\lambda I$, as using
$\lambda I - A$ would introduce lots of spurious minus signs.
\end{remark}

\begin{example}
\label{exa1.5}
Returning to the matrix $A =$
of Example~\ref{exa1.2}, we compute that 
$\lambda^2 - \lambda - 6 = (\lambda - 3)(\lambda + 2)$, so $A$ has eigenvalues
$3$ and $-2$.
Computation then shows that the eigenspace
$E_{3} = \ $
and that the eigenspace
$E_{-2} = \ $
\end{example}

Bye.
