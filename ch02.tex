\chapter{Applicative Patterns}
\label{ch02}

\section{Introduction}
\label{ch02.sec.1}

\newpage
\section{Guard}
\label{ch02.sec.2}

\subsection{Intent}
\label{ch02.sec.1.intent}
The Guard pattern serves as a protection mechanism for another resource, service or data source. Client requests are funneled through a resource to force a specific authentication and authorization strategy.

\subsection{Motivation}
\label{ch02.sec.1.motivation}

\subsection{Consequences}
\label{ch02.sec.1.consequences}

NetKernel?
OAuth 2?

\newpage
\section{Throttle}
\label{ch02.sec.3}

\subsection{Intent}
\label{ch02.sec.3.intent}
The Throttle pattern is a variation of the Guard pattern where the protection scheme involves scale rather than security. It is designed specifically to provide a resource to shape the request traffic into a predictable and sustainable load.

\subsection{Motivation}
\label{ch02.sec.3.motivation}

\subsection{Consequences}
\label{ch02.sec.3.consequences}

NetKernel Throttle in AWS

\newpage
\section{Overlay}
\label{ch02.sec.4}

\subsection{Intent}
\label{ch02.sec.4.intent}

\subsection{Motivation}
\label{ch02.sec.4.motivation}

\subsection{Consequences}
\label{ch02.sec.4.consequences}

\newpage
\section{Transformation}
\label{ch02.sec.5}

\subsection{Intent}
\label{ch02.sec.5.intent}
The Transformation pattern is a generalization over the approach of producing new resources to extract content from or transform the shape of content from another service or resource. As with the Named Query pattern, doing so can induce reusability and cacheability from sources that otherwise do not support such properties. It can also be used to add content negotiation to sources that do not support it.

\subsection{Motivation}
\label{ch02.sec.5.motivation}

\subsection{Consequences}
\label{ch02.sec.5.consequences}

RDFa Distiller

